\documentclass[11pt]{article}

\usepackage[includeheadfoot,margin=0.6cm,top=0.3cm,bottom=0.6cm,headsep=0.2cm]{geometry}

\usepackage{tabularx}
\usepackage[table]{xcolor}
\usepackage{multicol}
\usepackage{fontspec}
\usepackage{pgffor}
\usepackage{fancyhdr}
\usepackage{titlesec}
\usepackage{makecell}

\usepackage{hyperref}
% Hack to get url in blue with underline
\hypersetup{colorlinks,urlcolor=blue,urlbordercolor=blue}

\urlstyle{same}

\makeatletter
\Hy@AtBeginDocument{
  \def\@pdfborderstyle{/S/U/W 1}
}
\makeatother

% LaTeX counter interface for \rownum
\makeatletter
\@ifundefined{c@rownum}{%
  \let\c@rownum\rownum
}{}
\@ifundefined{therownum}{%
  \def\therownum{\@arabic\rownum}%
}{}
\makeatother

% Reduce vertical spacing before and after Special Rules/Psychic title
\titlespacing*{\subsubsection}{0pt}{1pt plus 1pt minus 1pt}{1pt plus
  1pt minus 1pt}

\setmainfont{Carlito}

% Remove page number
\pagenumbering{gobble}


\pagestyle{fancy}

\definecolor{lgrey}{rgb}{0.82, 0.82, 0.82}

\newcommand{\mytitle}[1]{
  \renewcommand{\headrulewidth}{0pt}
  \setlength{\headheight}{41 pt}
  \setlength{\parskip}{1 pt}

  % Add an extra thick white hline at end of table to have better
  % spacing between upgrade table
  \setlength{\arrayrulewidth}{3 pt}
  \arrayrulecolor{white}

  \chead{
    \LARGE \textbf{#1}\\
    \small by \textbf{Matthew Caron}
    (\footnotesize\url{https://www.mattcaron.net}\small)}

  % Suppress default right header value.
  \rhead{}
}

% Generate the table with all units and their stats.
% First parameter is the page number, for faction with more than 1 page.
\newcommand{\UnitTable}[1]{
  \centering
  \hyphenpenalty=100000
  \setlength\tabcolsep{2 pt}
  \rowcolors{1}{white}{lgrey}
  \footnotesize
  \begin{tabularx}{\linewidth}{lcc>{\hsize=1.4\hsize\linewidth=\hsize}X>{\hsize=.6\hsize\linewidth=\hsize}Xcc}
    \bf Name & \bf Qua & \bf Def & \bf Equipment & \bf Special Rules & \bf
    Upgrades & \bf Cost\\
    #1
  \end{tabularx}
}

% Generate the table for one upgrade group
\newcommand{\UpgradeTable}[1]{
  \hyphenpenalty=100000
  \setlength\tabcolsep{1 pt}
  \centering
  \footnotesize
  \rowcolors{1}{lgrey}{white}
  \begin{tabularx}{\linewidth}{Xlc}
    #1 \setcounter{rownum}{0} \\ \hline%
  \end{tabularx}
}

% Start a section with special rules
\newcommand{\specialrules}{
  \subsubsection*{Special Rules \hfill}
  \raggedright
  \footnotesize
}

% All special rules should use this function.
% First parameter is rule name.
% Second parameter is rule explanations.
\newcommand{\sprule}[2]{
  \textbf{#1:} #2

}

% Start a section with psychic spells
% #1 is list of spell, using \psychic
\newcommand{\startpsychic}[1]{
  \centering
  \subsubsection*{Psychic Spells \hfill}
  \raggedright
  \hyphenpenalty=100000
  \footnotesize
  \tabulinesep=2pt
  \setlength\tabcolsep{2 pt}
  \rowcolors{1}{lgrey}{white}
  \begin{tabularx}{\linewidth}{X}
    #1
  \end{tabularx}
}

% Psychic spell templates
% #1 is spell name
% #2 is spell difficulty
% #3 is spell description
\newcommand{\psychic}[3]{
  \textbf{#1 (#2):} #3 \\
}

\newcommand{\mysection}[1]{
  \section*{\centering #1}
  \raggedright
  \hrule
  \bigskip

}
\titlespacing{\section}{0pt}{*0}{*0}

% Start my new heading
% #1 is the heading.
\newcommand{\myheading}[1]{
  \subsubsection*{\centering #1}
  \raggedright
  \hrule
}

\renewenvironment{description}
  {\list{}{\labelwidth=0pt \leftmargin=0pt
   \let\makelabel\descriptionlabel}}
  {\endlist}

\newenvironment{mydescription}{
\begin{description}
  \setlength{\itemsep}{1pt}
  \setlength{\parskip}{0pt}
  \setlength{\parsep}{0pt}}{\end{description}
}

% Start my new heading
% #1 is the heading.
\newcommand{\myarmytitle}[1]{
  \begin{center}
  \Huge{\bf#1}
  \end{center}
  \hrule
}


\mytitle{6mm Modern v0.2}
\begin{document}

\begin{multicols*}{3}[]
\raggedcolumns

\mysection{Acknowledgements}

\begin{description}
\item {\bf Gaetano Ferrara}
  ({\footnotesize\url{https://onepagerules.com/}}) - for
  the One Page Rules project and points system.
\item {\bf Jocelyn Falempe}
  ({\footnotesize\url{https://github.com/kdj0c/onepagepoints}}) -
For the \LaTeX-fu. (I totally stole your templates).
\item {\bf Hasbro} for creating G.I. Joe. G.I. Joe is a registered trademark of
Hasbro, Inc. and is used without permission in a fan project. No infringement on
their intellectual property is intended - I just like the IP and want to fight
battles from the cartoon show.
\item {\bf Military service members worldwide} We play at this. You do it for
real. Respect.
\end{description}

\mysection{Crosslinks}

You can find 3D printable models for most of the army list entries in a separate
GitHub repository here:

\href{https://github.com/mattcaron/6mm_remixes}{https://github.com/\\mattcaron/6mm\_remixes}

\mysection{Notes}

\begin{description}
\item{\bf Basing} Use whatever basing you like (or none at all for vehicles). I
tend to use:
  \begin{itemize}
    \item 1" x 1" - Infantry bases.
    \item 1" x 1.5" - Light Tanks, IFVs, etc.
    \item 1" x 1.75" - Medium Tanks
    \item 1" x 2" - Heavy Tanks
  \end{itemize}
\end{description}

\mysection{Scale}
\begin{description}
\item {\bf Ground scale:} This game is designed to be played on a 4' x 4' table
or larger.  It is based off 12" = 1km, which ends up with 1" = 83m (not that
this is particularly relevant).
\item{\bf Model scale:} As the name implies, this is designed to be played with
standard 6mm / 1/285 / 1/300 models (often called Microarmor).
\end{description}

\mysection{Rules}

These rules are a modification of the OnePageRules Grimdark Future rules, which
can be downloaded at \url{https://onepagerules.com/portfolio/grimdark-future/},
with the following modifications:

\myheading{Models}

To be abundantly clear, in these rules a Model is an individual vehicle (based
or not) or a collectively based fighting element (squad, section, etc) of
infantry. As is true in the main rules, Models are grouped into Units.

\myheading{Movement}

Movement distances are as follows:

\begin{mydescription}

\item{\bf Advance:} The unit moves up to 4" and may shoot after moving.
\item{\bf Rush:} The unit moves up to 8" and may not shoot.
\item{\bf Charge:} The unit moves up to 8" to get into base contact with the
enemy.

\end{mydescription}

\myheading{Melee}

Melee combat is modified in that the Determine Attacks phase uses all weapons,
not just melee weapons. (The logic here is that, at this scale, it's more of a
close range firefight than a true hand to hand combat). All other melee combat
resolution steps remain unchanged.

\mysection{Special Rules}

The special rules are used as written, except as noted below:

\myheading{Aircraft}

Non-Aircraft models that shoot at Aircraft count as being an extra 6" away when
measuring and get –1 to their to hit rolls.

\smallskip

When an Aircraft is activated it must move a full 9" to 18" in a straight line.

\myheading{Ambush}

At the beginning of any round after the first you may place the model anywhere
on the table over 5" away from enemy units.

\myheading{Blast(X)}

Blast can deal more than one hit in the target unit if that unit is a stand of
Infantry.

\myheading{Deadly(X)}

Deadly cannot deal more than one hit in the target unit if that unit is a stand
of Infantry.

\myheading{Fast}

Fast units move 6" when using Advance actions and 12" when using Rush or Charge
actions.

\myheading{Hero}

Given the collective nature of stands in this game (a squad, a tank (and crew),
etc.) heroes are effectively upgrades to all models in a single unit - the idea
being that the hero's presence spurs the unit on to greater heroic feats, it is
a specialized unit led by that hero, or something similar. Note that you can
only have one of these heroes {\bf in the whole army}. (You can only have one
Snake Eyes).

The hero should be represented by a differently painted figure on a stand or
individual vehicle.

\myheading{Heroes and Combined Units}

If a player uses the Combined Units rule (buying two copies of a unit), both
units must pay the cost of the hero upgrade despite there only being one hero
vehicle / stand (this keeps the game fair).

\myheading{Infantry}

Models denoted as Infantry are Infantry. This is not a special rule, it just
means that they are more vulnerable to Blast(X) weapons (see Blast(X) rule,
above), and less vulnerable to Deadly(X) weapons (see Deadly(X) rule, above).

\myheading{Scout}

After all other units have been deployed models with scout may be deployed and
then moved by up to 6”, ignoring terrain.

\myheading{Slow}

Units with this special rule move 2" when using Advance actions and 4" when
using Rush or Charge actions.

\myheading{Transport}

... embarked units may use any action to disembark but only move up to 2".

Additionally, the stated transport capacity is the number of Infantry stands which can be carried.

\newpage

\mysection{Optional Rules}

All of the optional rules should work just fine as long as you halve the ranges
(round up), but the following rules are modified:

\myheading{Extra Actions: Overwatch}

The following changes are made to the Overwatch action:

\begin{itemize}

\item An individual model may hold an action; the entire unit does not have to
(this allows one vehicle in a unit to provide air defense while the others fire
for effect).
\item The shooting reaction distance is reduced to 12".

\end{itemize}

\myheading{Suppression}

Suppression is changed as follows:

\begin{itemize}
  \item Under {\bf Using Suppression}: Whenever a friendly unit within 3" is
  destroyed or routs after losing melee.
  \item Under {\bf Heroic Inspiration}: "and from all other friendly
  units within 6” (this doesn’t require any rolls)."
\end{itemize}

\mysection{Recommended Optional Rules}

The following optional rules are highly recommended:

\begin{itemize}
  \item Extra Actions: Overwatch
  \item Suppression
\end{itemize}

\mysection{Further Modifications}

\myheading{Splitting Fire}

As it is now very common for models to have multiple weapon systems (multiple
soldiers aggregated together, vehicles with multiple weapons, etc.) models may
now split fire between different target units as long as the split is based on
discrete weapons systems. You cannot split dice from the same machinegun across
two different targets, for example.

\mysection{Notational Changes}

\myheading{Basing}

In addition to the standard [X] nomenclature, I have prepended an (N)
nomenclature to entries. Given the following entry:

{\bf M1 Abrams (1)[4]}

This should be read as 4 tanks, 1 per base (a platoon). Given this one:

{\bf Fire Team (4)[2]}

That is to be read as 2 stands of infantry, 4 soldiers per base (a U.S. Army
Squad, comprised of 2 fire teams.). Conversely, a U.S. Marine squad of 3 bases
of 4 soldiers would look like this:

{\bf Fire Team (4)[3]}

\end{multicols*}

\newpage

\myarmytitle{G.I. Joe}

\UnitTable{

  Greenshirt Squad (4)[2] & 3+ & 5+ & \makecell[l]{Infantry Rifles (6",
    A4)} & Infantry, Slow, Tough(4) & A & 100pts\\
  A.P.C. Section (1)[2] & 3+ & 5+ & \makecell[l]{50mm cannon (24", A1, AP(3),
  Deadly(3))} & Fast, Tough(3), Transport(2) & & 215pts \\
  Dragonfly XH-1 Section (1)[2] & 3+ & 4+ &\makecell[l]{
  25mm chain gun (24", A3, AP(2))\\
  Sidewinder missiles (24", A1, Anti-Air, AP(3), Deadly(3)) \\
  Rocket pods (12", A2, Blast(3)) \\
  X-551 Mini cannons (12", A12) \\
  M-34 Grenade launcher (6", A2, Blast(3)) } & Fast, Flying, Tough(3) & B & 335pts \\
  F.L.A.K. Battery (3)[2] & 3+ & 5+ & \makecell[l]{F.L.A.K. cannon (12", A3,
    Anti-Air, Blast(3))} & Immobile, Infantry, Tough(3) & & 80pts\\
  H.A.L. Battery (2)[3] & 3+ & 5+ & \makecell[l]{Heavy Artillery Laser (18", A2,
    AP(4), Deadly(3))} & Immobile, Infantry, Tough(2) & C & 125pts\\  
  Headquarters Command Center & 3+ & 2+ & \makecell[l]{
  .50 Cal. MG (24", A3, AP(1)) \\
  .50 Cal. MG (24", A3, AP(1)) \\
  Twin Anti-Tank gun (48", A2, AP(4), Deadly(6)) \\
  Flamethrower (A6) \\
  } & Immobile, Tough(24) & & 1615pts\\ 
  M.M.S. Battery (2)[3] & 3+ & 5+ & \makecell[l]{Raytheon MIM-23 HAWK missiles
  \\ (24", A2, Anti-Air, AP(3), Deadly(3))} & Immobile, Infantry, Tough(2) & D & 175pts\\ 
  M.O.B.A.T Section (1)[2] & 3+ & 2+ & \makecell[l]{130mm Cannon (48", A1,
  AP(4), Deadly(6)) \\ .50 Cal. MG (24", A3, AP(1))} & Tough(9) & E & 680pts\\ 
  P.A.C./R.A.T. Unit (3)[1] & 5+ & 5+ & \makecell[l]{
    Quad Machineguns (12", A12) \\
    Flamethrower (A6) \\
    Missile Launcher (12", A1, Anti-Air, AP(3), Deadly(3))
  } & Slow, Tough(6) & & 105pts\\ 
  Polar Battle Bear Section (2)[2] & 3+ & 5+ & \makecell[l]{
    Twin Machineguns (12", A12) \\
    Rockets (12", A2, Blast(3)) } & Fast, Infantry, Tough(2) & & 160pts\\ 
  R.A.M. Section (2)[2] & 3+ & 5+ & \makecell[l]{Gatling Gun (12", A12)} & Fast,
  Infantry, Tough(2) & F & 130pts\\ 
  Skystriker XP-14F Flight & 3+ & 4+ & \makecell[l]{
  20mm Cannon (24", A6, AP(2), Deadly(2))\\
  Missiles (36", A1, Anti-Air, AP(3), Deadly(3))} & Aircraft, Tough(6) & G & 465pts\\
  V.A.M.P. Section (1)[2] & 3+ & 4+ & \makecell[l]{Twin Machine Guns (24", A6,
  AP(1))} & Fast, Scout, Tough(3) & H & 220pts\\ 
  Whirlwind Battery (2)[3] & 3+ & 5+ & Twin Gatling Gun (12", A24, Anti-Air) &
  Immobile, Infantry, Tough(2) & & 200pts \\
  Wolverine Section (1)[2] & 3+ & 3+ & ATGM (24", A1, AP(3), Deadly(3)) & Fast,
  Tough(6) & I & 290pts \\

}

\begin{multicols*}{3}[]

  \UpgradeTable{
    \multicolumn{2}{p{\dimexpr \linewidth - 2pt \relax}}
    {\bf A | Upgrade unit with one of: } & \\
      Hero - Airborne (Ambush) & +20pts \\
      Hero - Breaker (??) & +XXpts \\
      \makecell[l]{Hero - Flash (Replace Infantry Rifles with\\ Laser Rifles
      (6", A4, AP(4))} & +20pts \\
      Hero - Doc (Regeneration) & +60pts \\
      Hero - Duke (Quality 2+) & +35pts \\
      Hero - Grunt (Fearless) & +40pts \\
      Hero - Gung-Ho (Strider) & +10pts \\
      \makecell[l]{Hero - Rock n' Roll (Replace Infantry Rifles\\ with SAWs (6",
      A12))} & +20pts \\
      \makecell[l]{Hero - Scarlett (Melee Combat Training \\ (Add A4, Deadly(2),
      Stealth))} & +35pts \\
      Hero - Short-Fuze (Replace Infantry Rifles with Mortars (12", A2,
      Blast(3), Indirect)) & +30pts \\
      Hero - Snake-Eyes (Impact(4), Stealth) & +40pts \\
      Hero - Snowjob (Sniper) & +10pts \\
      Hero - Stalker (Scout) & +20pts \\
      Hero - Torpedo (??) & +XXpts \\
      Hero - Tripwire (Add Explosives(A1, AP(3), Deadly(3))) & +25pts \\
      \makecell[l]{Hero - Zap (Replace Infantry Rifles \\ with Anti-Tank Weapons
      \\ (6", A4, AP(3), Deadly(3)))} & +40pts \\
    {\bf Upgrade unit with one of:} & \\
       J.U.M.P. Jetpack (Flying) & +20pts \\
       Falcon Glider (Ambush) & +20pts}

  \UpgradeTable{
    \multicolumn{2}{p{\dimexpr \linewidth - 2pt \relax}}
    {\bf B | Upgrade unit with: } & \\
      Hero - Wild Bill (Relentless) & +25pts}

  \UpgradeTable{
    \multicolumn{2}{p{\dimexpr \linewidth - 2pt \relax}}
    {\bf C | Upgrade unit with: } & \\
      Hero - Grand Slam (Increase range to 24") & +40pts}

  \UpgradeTable{
    \multicolumn{2}{p{\dimexpr \linewidth - 2pt \relax}}
    {\bf D | Upgrade unit with: } & \\
      Hero - Hawk (Increase to AP(4)) & +85pts}

  \UpgradeTable{
    \multicolumn{2}{p{\dimexpr \linewidth - 2pt \relax}}
    {\bf E | Upgrade unit with: } & \\
      Hero - Steeler (Ambush) & +20pts} 
      
  \UpgradeTable{
    \multicolumn{2}{p{\dimexpr \linewidth - 2pt \relax}}
    {\bf F | Upgrade unit with: } & \\
    \makecell[l]{Hero - Rock n' Roll (Replace Gatling Gun \\ with High Speed
      Gatling Gun (12", A24))} & +55pts}      

  \UpgradeTable{
    \multicolumn{2}{p{\dimexpr \linewidth - 2pt \relax}}
    {\bf G | Upgrade unit with: } & \\
      Hero - Ace (Unit quality 2+) & +175pts}

  \UpgradeTable{
    \multicolumn{2}{p{\dimexpr \linewidth - 2pt \relax}}
    {\bf H | Upgrade unit with: } & \\
      Hero - Clutch (Strider) & +5pts}

  \UpgradeTable{
    \multicolumn{2}{p{\dimexpr \linewidth - 2pt \relax}}
    {\bf I | Upgrade unit with: } & \\
      Hero - Cover Girl (Regeneration) & +80pts}      

  \specialrules {
    {\bf Yo Joe!:} Once per game, the G.I. Joe player may invoke the battle cry
    "Yo Joe!" (yes, the player must actually say it in an awesome and dramatic
    fashion). For the remainder of the turn, the player may reroll all roles
    (just once).
  }
\end{multicols*}

\newpage

\myarmytitle{Cobra}

\UnitTable{
  Trooper Squad (5)[3] & 5+ & 6+ & \makecell[l]{Infantry Rifles (6",
  A5)} & Infantry, Slow, Tough(5) & A & 100pts\\
  F.A.N.G. Section (1)[2] & 5+ & 5+ & \makecell[l]{Missiles (24", A1, Anti-Air,
  AP(3), Deadly(3)), \\ 20mm Grenade Launcher (12", A2, Blast(3))\\ Bomb (6",
  A1, Blast(6))} & Fast, Flying, Tough(3) & & 125pts\\
  H.I.S.S. MkI Tank Section (1)[2] & 5+ & 3+ & ATGM system (42", A1, AP(3),
  Deadly(3)) & Fast, Tough(6) & B & 215pts\\
  H.I.S.S. MkI Transport Section (1)[2] & 5+ & 3+ & ATGM system (42", A1, AP(3), Deadly(3) & Fast, Tough(6), Transport(1) & C & 240pts\\
  S.N.A.K.E. Squad (3)[3] & 5+ & 4+ & \makecell[l]{Machineguns (6",
  A6), Flamethrowers (Adds Impact(3))} & Impact(3), Tough(6) &  & 250pts\\
}

\begin{multicols*}{3}[]
  \UpgradeTable{
    \multicolumn{2}{p{\dimexpr \linewidth - 2pt \relax}}
    {\bf A | Upgrade with one of: } & \\
    \makecell[l]{Hero - Cobra Commander\\ (Quality 4+, Fearless)} & +65pts \\
    \makecell[l]{Hero - Destro (Defense 5+)} & +30pts \\
    \makecell[l]{Hero - Major Bludd (Sniper)} & +25pts \\
    {\bf Upgrade with: } & \\
    Viper Glider (Ambush) & +10pts}


  \UpgradeTable{
    \multicolumn{2}{p{\dimexpr \linewidth - 2pt \relax}}
    {\bf B | Replace any ATGM with: } & \\
    \makecell[l]{Twin 90mm Cannon\\ (36", A2, AP(3), Deadly(3))} & +30pts \\
    {\bf Upgrade all: } & \\
    \makecell[l]{Drivers to Track-Vipers (Quality 4+)} & +40pts}


  \UpgradeTable{
    \multicolumn{2}{p{\dimexpr \linewidth - 2pt \relax}}
    {\bf C | Replace any ATGM with: } & \\
    \makecell[l]{Twin 30mm Cannon\\ (24", A4, AP(2), Deadly(3))} & +35pts \\
    {\bf Upgrade all: } & \\
    \makecell[l]{Drivers to Track-Vipers (Quality 4+)} & +40pts}

  \specialrules {
    {\bf Cobra!:} Once per game, the Cobra player may invoke the battle cry
    "Cobra!" (yes, the player must actually say it in an awesome and dramatic
    fashion). For the remainder of the turn, the player may reroll all roles
    (just once).
  }
\end{multicols*}
  
\end{document}