\documentclass[11pt]{article}

\usepackage[includeheadfoot,margin=0.6cm,top=0.3cm,bottom=0.6cm,headsep=0.2cm]{geometry}

\usepackage{tabu}
\usepackage[table]{xcolor}
\usepackage{multicol}
\usepackage{fontspec}
\usepackage{pgffor}
\usepackage{fancyhdr}
\usepackage{titlesec}
\usepackage{makecell}

\usepackage{hyperref}
% Hack to get url in blue with underline
\hypersetup{colorlinks,urlcolor=blue,urlbordercolor=blue}

\urlstyle{same}

\makeatletter
\Hy@AtBeginDocument{
  \def\@pdfborderstyle{/S/U/W 1}
}
\makeatother

% LaTeX counter interface for \rownum
\makeatletter
\@ifundefined{c@rownum}{%
  \let\c@rownum\rownum
}{}
\@ifundefined{therownum}{%
  \def\therownum{\@arabic\rownum}%
}{}
\makeatother

% Reduce vertical spacing before and after Special Rules/Psychic title
\titlespacing*{\subsubsection}{0pt}{1pt plus 1pt minus 1pt}{1pt plus
  1pt minus 1pt}

\setmainfont{Carlito}

% Remove page number
\pagenumbering{gobble}


\pagestyle{fancy}

\definecolor{lgrey}{rgb}{0.82, 0.82, 0.82}

\newcommand{\mytitle}[1]{
  \renewcommand{\headrulewidth}{0pt}
  \setlength{\headheight}{41 pt}
  \setlength{\parskip}{1 pt}

  % Add an extra thick white hline at end of table to have better
  % spacing between upgrade table
  \setlength{\arrayrulewidth}{3 pt}
  \arrayrulecolor{white}

  \chead{
    \LARGE \textbf{#1}\\
    \small by \textbf{Matthew Caron}
    (\footnotesize\url{https://www.mattcaron.net}\small)}

  % Suppress default right header value.
  \rhead{}
}

% Generate the table with all units and their stats.
% First parameter is the page number, for faction with more than 1 page.
\newcommand{\UnitTable}[1]{
  \centering
  \hyphenpenalty=100000
  \setlength\tabcolsep{2 pt}
  \rowcolors{1}{white}{lgrey}
  \footnotesize
  \begin{tabu} to \linewidth {lccclX[4l]X[3l]cc}
    \bf Name & \bf Qua& \bf Def& \bf Dam & \bf Speed &\bf Equipment& \bf Special Rules& \bf Upgrades& \bf Cost\\
    #1
  \end{tabu}
}

% Generate the table for one upgrade group
\newcommand{\UpgradeTable}[1]{
  \hyphenpenalty=100000
  \setlength\tabcolsep{1 pt}
  \centering
  \footnotesize
  \rowcolors{1}{lgrey}{white}
  \begin{tabu} to \linewidth {X[l]c}
    #1 \setcounter{rownum}{0} \\ \hline%
  \end{tabu}
}

% Start a section with special rules
\newcommand{\specialrules}{
  \subsubsection*{Special Rules \hfill}
  \raggedright
  \footnotesize
}

% All special rules should use this function.
% First parameter is rule name.
% Second parameter is rule explanations.
\newcommand{\sprule}[2]{
  \textbf{#1:} #2

}

% Start a section with psychic spells
% #1 is list of spell, using \psychic
\newcommand{\startpsychic}[1]{
  \centering
  \subsubsection*{Psychic Spells \hfill}
  \raggedright
  \hyphenpenalty=100000
  \footnotesize
  \tabulinesep=2pt
  \setlength\tabcolsep{2 pt}
  \rowcolors{1}{lgrey}{white}
  \begin{tabu} to \linewidth {X}
    #1
  \end{tabu}
}

% Psychic spell templates
% #1 is spell name
% #2 is spell difficulty
% #3 is spell description
\newcommand{\psychic}[3]{
  \textbf{#1 (#2):} #3 \\
}

\newcommand{\mysection}[1]{
  \section*{\centering #1}
  \raggedright
  \hrule
  \bigskip

}
\titlespacing{\section}{0pt}{*0}{*0}

% Start my new heading
% #1 is the heading.
\newcommand{\myheading}[1]{
  \subsubsection*{\centering #1}
  \raggedright
  \hrule
}

\renewenvironment{description}
  {\list{}{\labelwidth=0pt \leftmargin=0pt
   \let\makelabel\descriptionlabel}}
  {\endlist}

\newenvironment{mydescription}{
\begin{description}
  \setlength{\itemsep}{1pt}
  \setlength{\parskip}{0pt}
  \setlength{\parsep}{0pt}}{\end{description}
}

% Start my new heading
% #1 is the heading.
\newcommand{\myarmytitle}[1]{
  \begin{center}
  \Huge{\bf#1}
  \end{center}
  \hrule
}


\mytitle{TANKS! v0.1}
\begin{document}

\begin{multicols*}{2}[]
\raggedcolumns

\mysection{Acknowledgements}

\begin{description}
\item {\bf Gaetano Ferrara}
  ({\footnotesize\url{http://onepagerules.wordpress.com/}}) - for
  the One Page Rules project and points system.
\item {\bf Jocelyn Falempe}
  ({\footnotesize\url{https://github.com/kdj0c/onepagepoints}}) -
For the \LaTeX-fu. (I totally stole your templates).
\item {\bf Kirk Rowe} - For the editing and suggestions.
\end{description}

\mysection{Resources}

The following are sources of 3D printable models for which statistics
are provided.

\begin{description}
\item {\bf Duncan 'Shadow' Louca}
  ({\footnotesize\url{https://www.duncanshadow.com/}})
\item {\bf 3DWargaming} ({\footnotesize\url{https://3dwargaming.com/}})
\end{description}

\mysection{Notes}

\begin{description}
\item{\bf Mounts:} It’s assumed that all vehicles have turrets with a 360
  degree field of fire. While not always historically accurate, it
  keeps the game simple and fun. If it bothers you, have a house rule
  that turretless tanks may only shoot at what their hull is facing.
\item{\bf Infantry weapons:} This game is about tanks and other armored
  vehicles. Infantry weapons which can’t hurt tanks are flatly
  ignored, as are infantry themselves. All of the weapons and damage
  values are similarly rescaled as to ignore infantry weapons. As
  such, the units presented here are incompatible with Grimdark Future
  units.
\item{\bf Attacks:} Each weapon rolls a number of dice. These dice reflect
  the damage potential of the weapon. The number of successes
  determines just how solid the hit is. No successes means that you
  flat missed.
\item{\bf Aircraft:} Attack helicopters and other close support aircraft
  are basically flying tanks. While they’re not here now, they may be
  in the future.
\item{\bf Units:} Each model is an individual. They are not fielded in
  squadrons. Therefore, there is no coherency.

\columnbreak

\item{\bf Historical Accuracy:} There is very little, especially when you
  compare the modern tanks against the historical ones and the science
  fiction ones against any of them. Everything is compressed so you
  can throw them all out on the play area and have them shoot each
  other up.
\item{\bf Why is everyone’s quality 4?} Because it’s a 50/50 shot roughly
  corresponding to veteran units. For the first cut, this is just
  about the differences in the tanks, not necessarily the crews. As
  time progresses and things shake out and I add different weapon
  options, I’ll likely add different quality crews as well.
\item{\bf How do I get X added?} Email me at matt@mattcaron.net.
\end{description}

\mysection{Scale}
\begin{description}
\item {\bf Ground scale:} This game is designed to be played on either
  a tabletop at reduced scale or on large floor or outside space
  (yard/garden, sandbox, etc.) at a full scale. As such, a “unitless”
  scale is used. It is designed to have inches used for the tabletop
  and feet for the ground. Metric values can always be used if you
  adjust values appropriately.
\item{\bf Model scale:} Any model scale works as long as it doesn’t
  look silly. 1/48 (or even larger) works for the large scale, but
  something like 1/72 or even 1/285 works well for tabletop scale.
\end{description}

\end{multicols*}
\pagebreak
\begin{multicols*}{3}[]

\raggedcolumns

\myheading{General Principles}

\begin{mydescription}

\item{\bf The most important rule:} Whenever the rules are unclear or
  don’t seem quite right, use common sense and personal
  preference. Have fun.

\item{\bf Quality Tests:} Whenever you must take a quality test, roll
  one six-sided die trying to score the unit’s quality value or
  higher, which counts as a success.

\item{\bf Modifiers:} No modification will take a roll above a 6+. A 6
  always succeeds. A 1 always fails.

\item{\bf Line of sight:} If you can draw a straight line from
  attacker to target without passing through any obstacle or unit,
  then it has line of sight.

\end{mydescription}

\myheading{Playing the Game}

The game is played in rounds in which players alternate in activating
one unit each until all units have been activated. The player that
deployed first starts activating first on the first round. Each new
round the layer that finished activating first in the previous round
gets to go first.

\myheading{Activation}

The player decides which unit he activates, and it may do one of the
following:

\noindent 
\rowcolors{1}{white}{lgrey}
\begin{tabular}{lp{2cm}p{2cm}}
\bf{Action} & \bf{Move} & \bf{Notes} \\
Hold & 0 & May shoot \\
Advance & Slow: 4 \newline Normal: 6 \newline Fast: 9 & May shoot
                                                        after moving \\
Rush & Slow: 8 \newline Normal: 12 \newline Fast: 18 & May shoot
                                                        after moving \\
\end{tabular}

\myheading{Movement}

Models may move up to their given move value, making any number of
turns along the way. A model may end its movement in any orientation
(facing), but may not rotate once movement is completed.

\myheading{Shooting}

All models that are in range and have line of sight to an enemy model
may fire any or all of their weapons at it. All weapons have a 360
degree field of fire. Units may split fire by weapons, but a single
weapon’s dice pool may not be divided. If a unit splits fire, it must
predeclare which models are being shot with which weapon before any
dice are rolled.

Shooting models take one Quality test per attack and each success is a
hit. For each hit, the defender rolls one die trying to score the
target’s Defense value or higher, with each success causing a point of
damage. Record the damage by some convenient method (a die placed on
or near the model works well). Once the damage meets or exceeds the
unit’s damage rating, remove the unit from the field or otherwise
indicate its destruction.

\myheading{Ramming (Optional rule)}

If all players agree, a unit may make a Rush move that brings it into
contact with another unit. This is called a ram. On a ram, both units
make an attack on each other using a number of dice equal to their
damage stat, as the damage roughly represents the structural integrity
and mass of the unit. Note that BOTH attacker and defender get to make
this roll, and it may result in simultaneous unit destruction.

\myheading{Terrain}

\begin{mydescription}

\item{\bf Cover (forests, ruins, sandbags, etc.):} Units that shoot at
  enemy units mostly behind cover get -1 to their shooting rolls.

\item{\bf Difficult Terrain (wood, mud, rivers, etc.):} Units moving
  through difficult terrain may not make “rush” moves.

\item{\bf Dangerous Terrain (tank traps, mines, etc.):} Roll 1 die for
  every model that moves across dangerous terrain or activates in
  it. On a roll of 1, the model takes 1 damage.

\end{mydescription}

\myheading{Facing}

Tank armor is thickest at the front, so facing matters. Units
defending from an attack striking their side armor get -1 to their
defense rolls. Units defending from an attack striking the rear of an
enemy unit get -2 to their defense rolls.

\myheading{Weapons}

Weapon profiles are listed directly on the unit’s card and are
represented like this:

\begin{itemize}

\item Name (Range, Attacks, Special Rules)

\end{itemize}

\columnbreak

\myheading{Special Rules}

\begin{mydescription}

\item{\bf Anti-Air:} This weapon gets +1 to its shooting rolls against
  enemy Flyers.

\item{\bf Flying:} This model may move through other units and
  obstacles and it may ignore terrain effects.

\item{\bf Indirect:} This model may be fired at enemies that are not
  within line of sight. Targets not within line of sight count as
  being in cover.

\item{\bf Linked:} This weapon gets +1 to its attack rolls.

\item{\bf Scout:} This model is deployed after all other units have
  been deployed. Scout units may be placed anywhere over 12 away from
  enemies. If both players have scout units, the players must roll off
  to see who deploys first.

\item{\bf Sniper:} Models firing this weapon count as having Quality
  2+ and ignore cover.

\item{\bf Stealth:} Enemies get -1 to their shooting rolls against
  this unit.

\item{\bf Strider:} This model may make Rush moves through difficult
  terrain.

\end{mydescription}

\myheading{Hit Effects (Optional rule)}

If all players agree, every time a unit takes damage roll on the
following table for each damage source (that is, if you take 3 damage
from a single source, you roll once, not 3 times).

\noindent 
\rowcolors{1}{white}{lgrey}
\begin{tabular}{lp{5cm}}
\bf{Roll} & \bf{Effect} \\
6 & Ammo cooks off. Kaboom! Unit destroyed. \\
5 & Random weapon destroyed. If none left, unit destroyed. \\
4 & Immobilized for rest of game. If already immobilized, weapon
    destroyed. \\
3 & Stunned. May not move or shoot next turn. \\
2 & Shaken. May not shoot next turn. \\
1 & No additional effect.\\

\end{tabular}

\end{multicols*}

\newpage

\myarmytitle{Duncan's stuff}

\UnitTable{

  AFV & 4+ & 5+ & 1 & Fast & \makecell[l]{Linked Light Cannons(36,
    A2, Linked)} & Scout & & 34 pts\\

  Tank & 4+ & 3+ & 6 & Normal & \makecell[l]{Heavy Cannon(72,
    A6),\\Plasma Cannon(48, A4)} & & & 162 pts\\

  APC & 4+ & 5+ & 2 & Normal & \makecell[l]{Tribarrel Cannon(24,
    A1)} & & & 23 pts\\

  Self-Propelled Howitzer & 4+ & 6+ & 2 & Normal &
  \makecell[l]{Howitzer(72, A6, Indirect)} & & & 95 pts\\

  Rocket Artillery & 4+ & 5+ & 2 & Normal & \makecell[l]{Rocket
    Barrage(60, A6, Indirect)} & & &	88 pts\\

  Quad Laser & 4+ & 5+ & 3 & Normal & \makecell[l]{Quad Laser(48,
    A4)} & & & 54 pts\\

  Field Howitzer & 4+ & 6+ & 1 & Slow & \makecell[l]{Howitzer(72,
    A6, Indirect)} & & & 82 pts\\

  Field Mortar (Light) & 4+ & 6+ & 1 & Slow & \makecell[l]{Light
    Mortar(48, A3, Indirect}) & & & 28 pts\\

  Scout Walker & 4+ & 5+ & 1 & Fast & \makecell[l]{Plasma Cannon(48,
    A4)} & Scout & & 46 pts \\

  Field Mortar (Heavy) & 4+ & 6+ & 1 & Slow & \makecell[l]{Heavy
    Mortar(48, A6, Indirect)} & & & 55 pts\\

  Siege Tank & 4+ & 5+ & 3 & Slow & \makecell[l]{Siege Cannon(24, A9)}
  & & & 51 pts\\

  MKII AFV & 4+ & 5+ & 2 & Fast &  \makecell[l]{Linked Light Cannons(36,
    A2, Linked)} & Scout & & 44 pts\\

  MKII Light Tank & 4+ & 4+ & 3 & Normal & \makecell[l]{Medium
    Cannon(48, A4)} & & & 60 pts\\

  MKII Rocket Launcher (ATGM) & 4+ & 5+ & 3 & Fast &
  \makecell[l]{ATGM(60, A6)} & & & 81 pts\\

  Heavy Tank & 4+ & 2+ & 9 & Slow & \makecell[l]{Linked Heavy Cannon(72,
    A6, Linked)} & & & 228 pts\\

  Heavy Tank Tank Hunter & 4+ & 2+ & 9 & Slow & \makecell[l]{High
    Velocity Heavy Cannon(72, A9)} & & & 237 pts\\

}

\newpage

\myarmytitle{3D Wargaming's stuff}

\UnitTable{


  Formic AWC w/ ATGM & 4+ & 5+ & 2 & Normal & \makecell[l]{ATGM (60,
    A6)} & & & 65 pts\\

  Pillar MBT & 4+ & 3+ & 6 & Normal & \makecell[l]{Heavy Cannon (72, A6)} & &
  & 138 pts\\

  Guardian APC & 4+ & 5+ & 2 & Fast & \makecell[l]{Autocannon (36,
    A1)} & & & 31 pts\\

  Marauder Light Tank & 4+ & 4+ & 3 & Normal & \makecell[l]{Light
    Cannon (48, A4)} & & & 60 pts\\

  Lux Sentinel & 4+ & 4+ & 2 & Normal & \makecell[l]{Laser Cutter (24,
    A3)} & & & 33 pts\\

  Lux Occuli & 4+ & 3 & 1 & Fast & \makecell[l]{Laser Cutter (24,
    A3)} & & & 29 pts\\

  Gleam Occuli & 4+ & 3+ & 1 & Fast & \makecell[l]{Laser Blaster (12,
    A6)} & & & 29 pts\\

  Tiger Ausf E & 4+ & 3+ & 7 & Slow & \makecell[l]{8.8cm cannon (60,
    A5)} & & & 130 pts\\

  Panther Ausf D & 4+ & 4+ & 5 & Normal & \makecell[l]{7.5cm cannon
    (48, A4)} & & & 84 pts\\

  Panzer IV Ausf G & 4+ & 4+ & 4 & Normal & \makecell[l]{7.5 cm
    cannon (30, A4)} & & & 63 pts\\

  StuG IV & 4+ & 4+ & 4 & Normal & \makecell[l]{7.5 cm cannon (30,
    A4)} & & & 63 pts\\

  Su-100 & 4+ & 4+ & 4 & Normal & \makecell[l]{100mm cannon (60,
    A5)} & & & 86 pts\\

  T-34/76 & 4+ & 5+ & 4 & Normal & \makecell[l]{76mm cannon (30,
    A3)} & & & 51 pts\\

  T-34/85 & 4+ & 4+ & 4 & Normal & \makecell[l]{85mm cannon (48,
    A4)} & & & 72 pts\\
  
  Sherman & 4+ & 5+ & 3 & Normal & \makecell[l]{75mm cannon (30,
    A3)} & & & 41 pts\\

  Sherman Firefly & 4+ & 5+ & 3 & Normal & \makecell[l]{76.2mm cannon
    (48, A4)} & & & 54 pts\\

}
\end{document}
